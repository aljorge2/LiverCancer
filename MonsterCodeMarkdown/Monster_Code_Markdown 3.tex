% Options for packages loaded elsewhere
\PassOptionsToPackage{unicode}{hyperref}
\PassOptionsToPackage{hyphens}{url}
%
\documentclass[
]{article}
\title{Code Monster Markdown}
\author{}
\date{\vspace{-2.5em}}

\usepackage{amsmath,amssymb}
\usepackage{lmodern}
\usepackage{iftex}
\ifPDFTeX
  \usepackage[T1]{fontenc}
  \usepackage[utf8]{inputenc}
  \usepackage{textcomp} % provide euro and other symbols
\else % if luatex or xetex
  \usepackage{unicode-math}
  \defaultfontfeatures{Scale=MatchLowercase}
  \defaultfontfeatures[\rmfamily]{Ligatures=TeX,Scale=1}
\fi
% Use upquote if available, for straight quotes in verbatim environments
\IfFileExists{upquote.sty}{\usepackage{upquote}}{}
\IfFileExists{microtype.sty}{% use microtype if available
  \usepackage[]{microtype}
  \UseMicrotypeSet[protrusion]{basicmath} % disable protrusion for tt fonts
}{}
\makeatletter
\@ifundefined{KOMAClassName}{% if non-KOMA class
  \IfFileExists{parskip.sty}{%
    \usepackage{parskip}
  }{% else
    \setlength{\parindent}{0pt}
    \setlength{\parskip}{6pt plus 2pt minus 1pt}}
}{% if KOMA class
  \KOMAoptions{parskip=half}}
\makeatother
\usepackage{xcolor}
\IfFileExists{xurl.sty}{\usepackage{xurl}}{} % add URL line breaks if available
\IfFileExists{bookmark.sty}{\usepackage{bookmark}}{\usepackage{hyperref}}
\hypersetup{
  pdftitle={Code Monster Markdown},
  hidelinks,
  pdfcreator={LaTeX via pandoc}}
\urlstyle{same} % disable monospaced font for URLs
\usepackage[margin=1in]{geometry}
\usepackage{color}
\usepackage{fancyvrb}
\newcommand{\VerbBar}{|}
\newcommand{\VERB}{\Verb[commandchars=\\\{\}]}
\DefineVerbatimEnvironment{Highlighting}{Verbatim}{commandchars=\\\{\}}
% Add ',fontsize=\small' for more characters per line
\usepackage{framed}
\definecolor{shadecolor}{RGB}{248,248,248}
\newenvironment{Shaded}{\begin{snugshade}}{\end{snugshade}}
\newcommand{\AlertTok}[1]{\textcolor[rgb]{0.94,0.16,0.16}{#1}}
\newcommand{\AnnotationTok}[1]{\textcolor[rgb]{0.56,0.35,0.01}{\textbf{\textit{#1}}}}
\newcommand{\AttributeTok}[1]{\textcolor[rgb]{0.77,0.63,0.00}{#1}}
\newcommand{\BaseNTok}[1]{\textcolor[rgb]{0.00,0.00,0.81}{#1}}
\newcommand{\BuiltInTok}[1]{#1}
\newcommand{\CharTok}[1]{\textcolor[rgb]{0.31,0.60,0.02}{#1}}
\newcommand{\CommentTok}[1]{\textcolor[rgb]{0.56,0.35,0.01}{\textit{#1}}}
\newcommand{\CommentVarTok}[1]{\textcolor[rgb]{0.56,0.35,0.01}{\textbf{\textit{#1}}}}
\newcommand{\ConstantTok}[1]{\textcolor[rgb]{0.00,0.00,0.00}{#1}}
\newcommand{\ControlFlowTok}[1]{\textcolor[rgb]{0.13,0.29,0.53}{\textbf{#1}}}
\newcommand{\DataTypeTok}[1]{\textcolor[rgb]{0.13,0.29,0.53}{#1}}
\newcommand{\DecValTok}[1]{\textcolor[rgb]{0.00,0.00,0.81}{#1}}
\newcommand{\DocumentationTok}[1]{\textcolor[rgb]{0.56,0.35,0.01}{\textbf{\textit{#1}}}}
\newcommand{\ErrorTok}[1]{\textcolor[rgb]{0.64,0.00,0.00}{\textbf{#1}}}
\newcommand{\ExtensionTok}[1]{#1}
\newcommand{\FloatTok}[1]{\textcolor[rgb]{0.00,0.00,0.81}{#1}}
\newcommand{\FunctionTok}[1]{\textcolor[rgb]{0.00,0.00,0.00}{#1}}
\newcommand{\ImportTok}[1]{#1}
\newcommand{\InformationTok}[1]{\textcolor[rgb]{0.56,0.35,0.01}{\textbf{\textit{#1}}}}
\newcommand{\KeywordTok}[1]{\textcolor[rgb]{0.13,0.29,0.53}{\textbf{#1}}}
\newcommand{\NormalTok}[1]{#1}
\newcommand{\OperatorTok}[1]{\textcolor[rgb]{0.81,0.36,0.00}{\textbf{#1}}}
\newcommand{\OtherTok}[1]{\textcolor[rgb]{0.56,0.35,0.01}{#1}}
\newcommand{\PreprocessorTok}[1]{\textcolor[rgb]{0.56,0.35,0.01}{\textit{#1}}}
\newcommand{\RegionMarkerTok}[1]{#1}
\newcommand{\SpecialCharTok}[1]{\textcolor[rgb]{0.00,0.00,0.00}{#1}}
\newcommand{\SpecialStringTok}[1]{\textcolor[rgb]{0.31,0.60,0.02}{#1}}
\newcommand{\StringTok}[1]{\textcolor[rgb]{0.31,0.60,0.02}{#1}}
\newcommand{\VariableTok}[1]{\textcolor[rgb]{0.00,0.00,0.00}{#1}}
\newcommand{\VerbatimStringTok}[1]{\textcolor[rgb]{0.31,0.60,0.02}{#1}}
\newcommand{\WarningTok}[1]{\textcolor[rgb]{0.56,0.35,0.01}{\textbf{\textit{#1}}}}
\usepackage{graphicx}
\makeatletter
\def\maxwidth{\ifdim\Gin@nat@width>\linewidth\linewidth\else\Gin@nat@width\fi}
\def\maxheight{\ifdim\Gin@nat@height>\textheight\textheight\else\Gin@nat@height\fi}
\makeatother
% Scale images if necessary, so that they will not overflow the page
% margins by default, and it is still possible to overwrite the defaults
% using explicit options in \includegraphics[width, height, ...]{}
\setkeys{Gin}{width=\maxwidth,height=\maxheight,keepaspectratio}
% Set default figure placement to htbp
\makeatletter
\def\fps@figure{htbp}
\makeatother
\setlength{\emergencystretch}{3em} % prevent overfull lines
\providecommand{\tightlist}{%
  \setlength{\itemsep}{0pt}\setlength{\parskip}{0pt}}
\setcounter{secnumdepth}{-\maxdimen} % remove section numbering
\ifLuaTeX
  \usepackage{selnolig}  % disable illegal ligatures
\fi

\begin{document}
\maketitle

\hypertarget{markdown}{%
\subsection{'\,'\,'\,'markdown}\label{markdown}}

\hypertarget{r-markdown}{%
\section{\texorpdfstring{\textbf{R
Markdown}}{R Markdown}}\label{r-markdown}}

Title: ``Code Monster Markdown''\\
Author: Annika Jorgensen\\
Date: 02/02/2022\\
Purpose: This document is for the author to demonstrate understanding of
the ``DEG\_analysis\_changed\_comparison'' code and theory.

\hypertarget{libraries}{%
\subsubsection{\texorpdfstring{\textbf{Libraries}}{Libraries}}\label{libraries}}

The first chunk of code is dedicated to installing the libraries. These
libraries are to help execute the differential analysis and helps
visualize the data.

\begin{Shaded}
\begin{Highlighting}[]
\CommentTok{\#==================================}
\CommentTok{\# Markdown Librariese }
\CommentTok{\#==================================}

\CommentTok{\#==================================}
\CommentTok{\# DEG Analysis Librariese }
\CommentTok{\#==================================}

\FunctionTok{library}\NormalTok{(RColorBrewer)}
\FunctionTok{library}\NormalTok{(matrixStats)}
\FunctionTok{library}\NormalTok{(ggplot2)}
\FunctionTok{library}\NormalTok{(edgeR)}
\end{Highlighting}
\end{Shaded}

\begin{verbatim}
## Loading required package: limma
\end{verbatim}

\begin{Shaded}
\begin{Highlighting}[]
\FunctionTok{library}\NormalTok{(DESeq2)}
\end{Highlighting}
\end{Shaded}

\begin{verbatim}
## Loading required package: S4Vectors
\end{verbatim}

\begin{verbatim}
## Loading required package: stats4
\end{verbatim}

\begin{verbatim}
## Loading required package: BiocGenerics
\end{verbatim}

\begin{verbatim}
## 
## Attaching package: 'BiocGenerics'
\end{verbatim}

\begin{verbatim}
## The following object is masked from 'package:limma':
## 
##     plotMA
\end{verbatim}

\begin{verbatim}
## The following objects are masked from 'package:stats':
## 
##     IQR, mad, sd, var, xtabs
\end{verbatim}

\begin{verbatim}
## The following objects are masked from 'package:base':
## 
##     anyDuplicated, append, as.data.frame, basename, cbind, colnames,
##     dirname, do.call, duplicated, eval, evalq, Filter, Find, get, grep,
##     grepl, intersect, is.unsorted, lapply, Map, mapply, match, mget,
##     order, paste, pmax, pmax.int, pmin, pmin.int, Position, rank,
##     rbind, Reduce, rownames, sapply, setdiff, sort, table, tapply,
##     union, unique, unsplit, which.max, which.min
\end{verbatim}

\begin{verbatim}
## 
## Attaching package: 'S4Vectors'
\end{verbatim}

\begin{verbatim}
## The following objects are masked from 'package:base':
## 
##     expand.grid, I, unname
\end{verbatim}

\begin{verbatim}
## Loading required package: IRanges
\end{verbatim}

\begin{verbatim}
## 
## Attaching package: 'IRanges'
\end{verbatim}

\begin{verbatim}
## The following object is masked from 'package:grDevices':
## 
##     windows
\end{verbatim}

\begin{verbatim}
## Loading required package: GenomicRanges
\end{verbatim}

\begin{verbatim}
## Loading required package: GenomeInfoDb
\end{verbatim}

\begin{verbatim}
## Loading required package: SummarizedExperiment
\end{verbatim}

\begin{verbatim}
## Loading required package: MatrixGenerics
\end{verbatim}

\begin{verbatim}
## 
## Attaching package: 'MatrixGenerics'
\end{verbatim}

\begin{verbatim}
## The following objects are masked from 'package:matrixStats':
## 
##     colAlls, colAnyNAs, colAnys, colAvgsPerRowSet, colCollapse,
##     colCounts, colCummaxs, colCummins, colCumprods, colCumsums,
##     colDiffs, colIQRDiffs, colIQRs, colLogSumExps, colMadDiffs,
##     colMads, colMaxs, colMeans2, colMedians, colMins, colOrderStats,
##     colProds, colQuantiles, colRanges, colRanks, colSdDiffs, colSds,
##     colSums2, colTabulates, colVarDiffs, colVars, colWeightedMads,
##     colWeightedMeans, colWeightedMedians, colWeightedSds,
##     colWeightedVars, rowAlls, rowAnyNAs, rowAnys, rowAvgsPerColSet,
##     rowCollapse, rowCounts, rowCummaxs, rowCummins, rowCumprods,
##     rowCumsums, rowDiffs, rowIQRDiffs, rowIQRs, rowLogSumExps,
##     rowMadDiffs, rowMads, rowMaxs, rowMeans2, rowMedians, rowMins,
##     rowOrderStats, rowProds, rowQuantiles, rowRanges, rowRanks,
##     rowSdDiffs, rowSds, rowSums2, rowTabulates, rowVarDiffs, rowVars,
##     rowWeightedMads, rowWeightedMeans, rowWeightedMedians,
##     rowWeightedSds, rowWeightedVars
\end{verbatim}

\begin{verbatim}
## Loading required package: Biobase
\end{verbatim}

\begin{verbatim}
## Welcome to Bioconductor
## 
##     Vignettes contain introductory material; view with
##     'browseVignettes()'. To cite Bioconductor, see
##     'citation("Biobase")', and for packages 'citation("pkgname")'.
\end{verbatim}

\begin{verbatim}
## 
## Attaching package: 'Biobase'
\end{verbatim}

\begin{verbatim}
## The following object is masked from 'package:MatrixGenerics':
## 
##     rowMedians
\end{verbatim}

\begin{verbatim}
## The following objects are masked from 'package:matrixStats':
## 
##     anyMissing, rowMedians
\end{verbatim}

\begin{Shaded}
\begin{Highlighting}[]
\FunctionTok{library}\NormalTok{(limma)}
\FunctionTok{library}\NormalTok{(doParallel)}
\end{Highlighting}
\end{Shaded}

\begin{verbatim}
## Loading required package: foreach
\end{verbatim}

\begin{verbatim}
## Loading required package: iterators
\end{verbatim}

\begin{verbatim}
## Loading required package: parallel
\end{verbatim}

\begin{Shaded}
\begin{Highlighting}[]
\FunctionTok{library}\NormalTok{(variancePartition)}
\end{Highlighting}
\end{Shaded}

\begin{verbatim}
## Loading required package: BiocParallel
\end{verbatim}

\begin{verbatim}
## 
## Attaching package: 'variancePartition'
\end{verbatim}

\begin{verbatim}
## The following object is masked from 'package:limma':
## 
##     classifyTestsF
\end{verbatim}

\begin{Shaded}
\begin{Highlighting}[]
\FunctionTok{library}\NormalTok{(org.Hs.eg.db)}
\end{Highlighting}
\end{Shaded}

\begin{verbatim}
## Loading required package: AnnotationDbi
\end{verbatim}

\begin{verbatim}
## 
\end{verbatim}

\begin{Shaded}
\begin{Highlighting}[]
\FunctionTok{library}\NormalTok{(clusterProfiler)}
\end{Highlighting}
\end{Shaded}

\begin{verbatim}
## 
\end{verbatim}

\begin{verbatim}
## clusterProfiler v4.2.2  For help: https://yulab-smu.top/biomedical-knowledge-mining-book/
## 
## If you use clusterProfiler in published research, please cite:
## T Wu, E Hu, S Xu, M Chen, P Guo, Z Dai, T Feng, L Zhou, W Tang, L Zhan, X Fu, S Liu, X Bo, and G Yu. clusterProfiler 4.0: A universal enrichment tool for interpreting omics data. The Innovation. 2021, 2(3):100141
\end{verbatim}

\begin{verbatim}
## 
## Attaching package: 'clusterProfiler'
\end{verbatim}

\begin{verbatim}
## The following object is masked from 'package:AnnotationDbi':
## 
##     select
\end{verbatim}

\begin{verbatim}
## The following object is masked from 'package:IRanges':
## 
##     slice
\end{verbatim}

\begin{verbatim}
## The following object is masked from 'package:S4Vectors':
## 
##     rename
\end{verbatim}

\begin{verbatim}
## The following object is masked from 'package:stats':
## 
##     filter
\end{verbatim}

\begin{Shaded}
\begin{Highlighting}[]
\FunctionTok{library}\NormalTok{(GOSemSim)}
\end{Highlighting}
\end{Shaded}

\begin{verbatim}
## GOSemSim v2.20.0  For help: https://yulab-smu.top/biomedical-knowledge-mining-book/
## 
## If you use GOSemSim in published research, please cite:
## - Guangchuang Yu. Gene Ontology Semantic Similarity Analysis Using GOSemSim. In: Kidder B. (eds) Stem Cell Transcriptional Networks. Methods in Molecular Biology, 2020, 2117:207-215. Humana, New York, NY. doi:10.1007/978-1-0716-0301-7_11
## - Guangchuang Yu, Fei Li, Yide Qin, Xiaochen Bo, Yibo Wu, Shengqi Wang. GOSemSim: an R package for measuring semantic similarity among GO terms and gene products Bioinformatics 2010, 26(7):976-978. doi:10.1093/bioinformatics/btq064
\end{verbatim}

\begin{Shaded}
\begin{Highlighting}[]
\FunctionTok{library}\NormalTok{(biomaRt)}
\FunctionTok{library}\NormalTok{(UpSetR)}
\FunctionTok{library}\NormalTok{(VennDiagram)}
\end{Highlighting}
\end{Shaded}

\begin{verbatim}
## Loading required package: grid
\end{verbatim}

\begin{verbatim}
## Loading required package: futile.logger
\end{verbatim}

\begin{Shaded}
\begin{Highlighting}[]
\FunctionTok{library}\NormalTok{(ggrepel)}
\FunctionTok{library}\NormalTok{(dplyr)}
\end{Highlighting}
\end{Shaded}

\begin{verbatim}
## 
## Attaching package: 'dplyr'
\end{verbatim}

\begin{verbatim}
## The following object is masked from 'package:biomaRt':
## 
##     select
\end{verbatim}

\begin{verbatim}
## The following object is masked from 'package:AnnotationDbi':
## 
##     select
\end{verbatim}

\begin{verbatim}
## The following object is masked from 'package:Biobase':
## 
##     combine
\end{verbatim}

\begin{verbatim}
## The following objects are masked from 'package:GenomicRanges':
## 
##     intersect, setdiff, union
\end{verbatim}

\begin{verbatim}
## The following object is masked from 'package:GenomeInfoDb':
## 
##     intersect
\end{verbatim}

\begin{verbatim}
## The following objects are masked from 'package:IRanges':
## 
##     collapse, desc, intersect, setdiff, slice, union
\end{verbatim}

\begin{verbatim}
## The following objects are masked from 'package:S4Vectors':
## 
##     first, intersect, rename, setdiff, setequal, union
\end{verbatim}

\begin{verbatim}
## The following objects are masked from 'package:BiocGenerics':
## 
##     combine, intersect, setdiff, union
\end{verbatim}

\begin{verbatim}
## The following object is masked from 'package:matrixStats':
## 
##     count
\end{verbatim}

\begin{verbatim}
## The following objects are masked from 'package:stats':
## 
##     filter, lag
\end{verbatim}

\begin{verbatim}
## The following objects are masked from 'package:base':
## 
##     intersect, setdiff, setequal, union
\end{verbatim}

\begin{Shaded}
\begin{Highlighting}[]
\FunctionTok{library}\NormalTok{(stringr)}
\FunctionTok{library}\NormalTok{(forcats)}

\FunctionTok{setwd}\NormalTok{(}\StringTok{\textquotesingle{}\textasciitilde{}/R\textquotesingle{}}\NormalTok{)}
\end{Highlighting}
\end{Shaded}

\hypertarget{environment-parameters}{%
\subsection{\texorpdfstring{\textbf{Environment
parameters}}{Environment parameters}}\label{environment-parameters}}

This next section of code is dedicated to the environmental parameters.
Environmental parameters are a series of variables and other code that
will help make the rest of the script be easier to make and run later
on.

\hypertarget{working-directory}{%
\subsubsection{\texorpdfstring{\textbf{Working
Directory}}{Working Directory}}\label{working-directory}}

A working directory is a code that iterates a file path on your computer
th.t sets where the default location of any files that you read into R.
Working directories work different in R files than R Markdowns. R
Markdown files require directories to be defined at the end of each code
chunk. Meaning from here on out you will see working directories being
defined at the end of each code chunk.

\begin{Shaded}
\begin{Highlighting}[]
\FunctionTok{setwd}\NormalTok{(}\StringTok{\textquotesingle{}\textasciitilde{}/R\textquotesingle{}}\NormalTok{)}
\end{Highlighting}
\end{Shaded}

\hypertarget{defining-colors}{%
\subsubsection{\texorpdfstring{\textbf{Defining
Colors}}{Defining Colors}}\label{defining-colors}}

This chunk defines color palette variables that are going to be used in
plots later on the script. These variables are defined by conversiting
BrewerCode palettes into palettes that can be used in R.

\begin{Shaded}
\begin{Highlighting}[]
\NormalTok{viralPalette }\OtherTok{\textless{}{-}} \FunctionTok{brewer.pal}\NormalTok{(}\DecValTok{8}\NormalTok{, }\StringTok{"Set1"}\NormalTok{)}
\NormalTok{hbvColor }\OtherTok{\textless{}{-}}\NormalTok{ viralPalette[}\DecValTok{1}\NormalTok{]}
\NormalTok{hcvColor }\OtherTok{\textless{}{-}}\NormalTok{ viralPalette[}\DecValTok{2}\NormalTok{]}
\NormalTok{bothColor }\OtherTok{\textless{}{-}}\NormalTok{ viralPalette[}\DecValTok{3}\NormalTok{]}
\NormalTok{neitherColor }\OtherTok{\textless{}{-}}\NormalTok{ viralPalette[}\DecValTok{4}\NormalTok{]}

\NormalTok{sexTissuePalette }\OtherTok{\textless{}{-}} \FunctionTok{brewer.pal}\NormalTok{(}\DecValTok{12}\NormalTok{, }\StringTok{"Paired"}\NormalTok{)}
\NormalTok{maleTumorColor }\OtherTok{\textless{}{-}}\NormalTok{ sexTissuePalette[}\DecValTok{4}\NormalTok{]}
\NormalTok{maleAdjacentColor }\OtherTok{\textless{}{-}}\NormalTok{ sexTissuePalette[}\DecValTok{3}\NormalTok{]}
\NormalTok{femaleTumorColor }\OtherTok{\textless{}{-}}\NormalTok{ sexTissuePalette[}\DecValTok{6}\NormalTok{]}
\NormalTok{femaleAdjacentColor }\OtherTok{\textless{}{-}}\NormalTok{ sexTissuePalette[}\DecValTok{5}\NormalTok{]}
\FunctionTok{setwd}\NormalTok{(}\StringTok{\textquotesingle{}\textasciitilde{}/R\textquotesingle{}}\NormalTok{)}
\end{Highlighting}
\end{Shaded}

\hypertarget{read-in-data}{%
\subsubsection{\texorpdfstring{\textbf{Read in
data}}{Read in data}}\label{read-in-data}}

This code is where you read in all the data files that are going to be
used in the script. The data is also converted into a variety of
variables that makes the data easier to handle. The data is also cleaned
up to make sure the analysis done later is accurate and precise.

\begin{Shaded}
\begin{Highlighting}[]
\NormalTok{metadata }\OtherTok{\textless{}{-}} \FunctionTok{read.table}\NormalTok{(}\StringTok{"metadata\_for\_de.csv"}\NormalTok{, }\AttributeTok{row.names=}\DecValTok{1}\NormalTok{,}\AttributeTok{header=}\ConstantTok{TRUE}\NormalTok{, }\AttributeTok{sep=}\StringTok{","}\NormalTok{) }\CommentTok{\#changing the name of the file}
\NormalTok{tumorAdjacentExp }\OtherTok{\textless{}{-}} \FunctionTok{read.table}\NormalTok{(}\StringTok{"japan\_all\_samples\_salmon\_expression\_counts.txt"}\NormalTok{, }\AttributeTok{row.names =} \DecValTok{1}\NormalTok{, }\AttributeTok{header=}\ConstantTok{TRUE}\NormalTok{) }\CommentTok{\#changing the name of the file }
\FunctionTok{colnames}\NormalTok{(tumorAdjacentExp) }\OtherTok{\textless{}{-}} \FunctionTok{gsub}\NormalTok{(}\StringTok{"}\SpecialCharTok{\textbackslash{}\textbackslash{}}\StringTok{."}\NormalTok{, }\StringTok{"{-}"}\NormalTok{, }\FunctionTok{colnames}\NormalTok{(tumorAdjacentExp)) }\CommentTok{\#changing the column names }
\end{Highlighting}
\end{Shaded}

This next code chunk is very similar However, it does calculate
\textbf{gene length} which is the done by first defining a variable
named ``gene'''' and then changing the data type to a data frame. You
then redefine ``tumorAdjacentExp'' (defined above) to have the rows of
the previous ``tumorAdjacentExp'' and then the columns of ``GENEID''''
that lies within ``gene''.\\
Gene length is then defined to have ``with'' of genes in the rows and
`end-start' as a column

\begin{Shaded}
\begin{Highlighting}[]
\NormalTok{genes }\OtherTok{\textless{}{-}} \FunctionTok{read.table}\NormalTok{(}\StringTok{"gencodeTranscripts.txt"}\NormalTok{, }\AttributeTok{header=}\ConstantTok{TRUE}\NormalTok{, }\AttributeTok{sep=}\StringTok{"}\SpecialCharTok{\textbackslash{}t}\StringTok{"}\NormalTok{)}
\NormalTok{genes }\OtherTok{\textless{}{-}} \FunctionTok{data.frame}\NormalTok{(genes)}
\NormalTok{tumorAdjacentExp }\OtherTok{\textless{}{-}}\NormalTok{ tumorAdjacentExp[}\FunctionTok{rownames}\NormalTok{(tumorAdjacentExp) }\SpecialCharTok{\%in\%}\NormalTok{ genes}\SpecialCharTok{$}\NormalTok{GENEID ,]}
\NormalTok{genes }\OtherTok{\textless{}{-}}\NormalTok{ genes[}\FunctionTok{match}\NormalTok{(}\FunctionTok{rownames}\NormalTok{(tumorAdjacentExp), genes}\SpecialCharTok{$}\NormalTok{GENEID),]}
\CommentTok{\# Calculating gene length, this is needed for calculating the FPKM values}
\NormalTok{genes}\SpecialCharTok{$}\NormalTok{length }\OtherTok{\textless{}{-}} \FunctionTok{with}\NormalTok{(genes, end }\SpecialCharTok{{-}}\NormalTok{ start)}
\end{Highlighting}
\end{Shaded}

The next line shows a sample being removed due to low quality.

\begin{Shaded}
\begin{Highlighting}[]
\NormalTok{metadata}\OtherTok{\textless{}{-}}\NormalTok{metadata[}\SpecialCharTok{!}\NormalTok{(metadata}\SpecialCharTok{$}\NormalTok{ID }\SpecialCharTok{==} \StringTok{"RK023"}\NormalTok{), ]}
\end{Highlighting}
\end{Shaded}

This next chunk of data is dedicated to sub-setting and organizing the
data to make it easier to use going forward. Sub-setting means that the
data is being organized to match a count matrix. In this specific case
the count matrix is the sample ID attached to the tumors.

\begin{Shaded}
\begin{Highlighting}[]
\NormalTok{tumorAdjacentExpSubset }\OtherTok{\textless{}{-}}\NormalTok{ tumorAdjacentExp[,}\FunctionTok{colnames}\NormalTok{(tumorAdjacentExp) }\SpecialCharTok{\%in\%}\NormalTok{ metadata}\SpecialCharTok{$}\NormalTok{sampleid]}
\NormalTok{metadataSubset }\OtherTok{\textless{}{-}}\NormalTok{ metadata[metadata}\SpecialCharTok{$}\NormalTok{sampleid }\SpecialCharTok{\%in\%} \FunctionTok{colnames}\NormalTok{(tumorAdjacentExpSubset),]}
\NormalTok{metadataSubset }\OtherTok{\textless{}{-}}\NormalTok{ metadataSubset[}\FunctionTok{match}\NormalTok{(}\FunctionTok{colnames}\NormalTok{(tumorAdjacentExpSubset), metadataSubset}\SpecialCharTok{$}\NormalTok{sampleid),]}
\FunctionTok{rownames}\NormalTok{(metadataSubset) }\OtherTok{\textless{}{-}}\NormalTok{ metadataSubset}\SpecialCharTok{$}\NormalTok{sampleid}
\end{Highlighting}
\end{Shaded}

This next chunk of data is taking the meta data and subsetting it in
such a way that converts a series of categorical variables into factors.
This data also adds a tissue type.

\begin{Shaded}
\begin{Highlighting}[]
\NormalTok{metadataSubset}\SpecialCharTok{$}\NormalTok{tumor }\OtherTok{\textless{}{-}} \FunctionTok{as.numeric}\NormalTok{(}\FunctionTok{grepl}\NormalTok{(}\StringTok{\textquotesingle{}tumor\textquotesingle{}}\NormalTok{, metadataSubset}\SpecialCharTok{$}\NormalTok{sampleid, }\AttributeTok{ignore.case=}\NormalTok{T))}
\NormalTok{metadataSubset}\SpecialCharTok{$}\NormalTok{gender\_tissue }\OtherTok{\textless{}{-}} \FunctionTok{paste}\NormalTok{(metadataSubset}\SpecialCharTok{$}\NormalTok{Gender, metadataSubset}\SpecialCharTok{$}\NormalTok{tumor, }\AttributeTok{sep=}\StringTok{"\_"}\NormalTok{)}
\NormalTok{metadataSubset}\SpecialCharTok{$}\NormalTok{gender\_tissue\_viral }\OtherTok{\textless{}{-}} \FunctionTok{paste}\NormalTok{(metadataSubset}\SpecialCharTok{$}\NormalTok{gender\_tissue, metadataSubset}\SpecialCharTok{$}\NormalTok{Virus\_infection, }\AttributeTok{sep=}\StringTok{"\_"}\NormalTok{)}
\NormalTok{metadataSubset}\SpecialCharTok{$}\NormalTok{library\_type }\OtherTok{\textless{}{-}}\NormalTok{ metadataSubset}\SpecialCharTok{$}\NormalTok{strandedness}
\NormalTok{metadataSubset}\SpecialCharTok{$}\NormalTok{library\_type }\OtherTok{\textless{}{-}} \FunctionTok{factor}\NormalTok{(metadataSubset}\SpecialCharTok{$}\NormalTok{library\_type)}
\NormalTok{metadataSubset}\SpecialCharTok{$}\NormalTok{tumor }\OtherTok{\textless{}{-}} \FunctionTok{factor}\NormalTok{(metadataSubset}\SpecialCharTok{$}\NormalTok{tumor)}
\NormalTok{metadataSubset}\SpecialCharTok{$}\NormalTok{Ta }\OtherTok{\textless{}{-}} \FunctionTok{factor}\NormalTok{(metadataSubset}\SpecialCharTok{$}\NormalTok{Ta)}
\NormalTok{metadataSubset}\SpecialCharTok{$}\NormalTok{Portal\_vein\_invasion }\OtherTok{\textless{}{-}} \FunctionTok{factor}\NormalTok{(metadataSubset}\SpecialCharTok{$}\NormalTok{Portal\_vein\_invasion)}
\NormalTok{metadataSubset}\SpecialCharTok{$}\NormalTok{Hepatic\_vein\_invasion }\OtherTok{\textless{}{-}} \FunctionTok{factor}\NormalTok{(metadataSubset}\SpecialCharTok{$}\NormalTok{Hepatic\_vein\_invasion)}
\NormalTok{metadataSubset}\SpecialCharTok{$}\NormalTok{Bile\_duct\_invasion }\OtherTok{\textless{}{-}} \FunctionTok{factor}\NormalTok{(metadataSubset}\SpecialCharTok{$}\NormalTok{Bile\_duct\_invasion)}
\NormalTok{metadataSubset}\SpecialCharTok{$}\NormalTok{Liver\_fibrosisc }\OtherTok{\textless{}{-}} \FunctionTok{factor}\NormalTok{(metadataSubset}\SpecialCharTok{$}\NormalTok{Liver\_fibrosisc)}
\NormalTok{metadataSubset}\SpecialCharTok{$}\NormalTok{Prognosis }\OtherTok{\textless{}{-}} \FunctionTok{factor}\NormalTok{(metadataSubset}\SpecialCharTok{$}\NormalTok{Prognosis)}
\end{Highlighting}
\end{Shaded}


\end{document}
